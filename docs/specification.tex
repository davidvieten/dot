\documentclass[10pt]{article}

% Lines beginning with the percent sign are comments
% This file has been commented to help you understand more about LaTeX

% DO NOT EDIT THE LINES BETWEEN THE TWO LONG HORIZONTAL LINES

%---------------------------------------------------------------------------------------------------------

% Packages add extra functionality.
\usepackage{times,graphicx,epstopdf,fancyhdr,amsfonts,amsthm,amsmath,algorithm,algorithmic,xspace,hyperref}
\usepackage[left=1in,top=1in,right=1in,bottom=1in]{geometry}
\usepackage{sect sty}	%For centering section headings
\usepackage{enumerate}	%Allows more labeling options for enumerate environments 
\usepackage{epsfig}
\usepackage[space]{grffile}
\usepackage{booktabs}
\usepackage{forest}
\usepackage{enumitem}   
\usepackage{fancyvrb}
\usepackage{todonotes}

% This will set LaTeX to look for figures in the same directory as the .tex file
\graphicspath{.} % The dot means current directory.

\pagestyle{fancy}

\lhead{Final Project}
\rhead{\today}
\lfoot{CSCI 334: Principles of Programming Languages}
\cfoot{\thepage}
\rfoot{Spring 2024}

% Some commands for changing header and footer format
\renewcommand{\headrulewidth}{0.4pt}
\renewcommand{\headwidth}{\textwidth}
\renewcommand{\footrulewidth}{0.4pt}

% These let you use common environments
\newtheorem{claim}{Claim}
\newtheorem{definition}{Definition}
\newtheorem{theorem}{Theorem}
\newtheorem{lemma}{Lemma}
\newtheorem{observation}{Observation}
\newtheorem{question}{Question}

\setlength{\parindent}{0cm}

%---------------------------------------------------------------------------------------------------------

% DON'T CHANGE ANYTHING ABOVE HERE

% Edit below as instructed

\title{Dot Language Specification} % Replace SnappyLanguageName with your project's name

\author{David Vieten} % Replace these with real partner names.

\begin{document}
  
\maketitle

\subsection*{Introduction}
{  Fortunately, I have had the privelage of being around the sport of 
hockey for the majority of my lifetime. Over the years, it has become clear to me that the 
current method of drawing face-off plays is outdated. A face-off takes place starting a game or
following a whistle.  The ref drops the puck and two center-men battle for the puck.  Face-offs
play in important role in hockey, as it is an oppurtunity to gain possesion of the puck. As a result
coaches draw up routes for each of the 5 guys on the each, dictating what each player should do 
on a face-off win.  Each team often has multiple of these set routes called "face-off plays." These
faceoff plays may change on any given week depending on who the opponent is and what their weaknesses
are. As a result, the players are responsible for memorizing their routes.  Currently, coaches either draw
up the faceoff plays on a white board and hope the players can remember it, or they draw them up on paper
and distribute photo copies for the players to study. 

The programming language "Dot" solves this problem. "Dot" allows coaches to create face-off plays in an svg 
that can be emailed to players.  This language saves coaches a headache by providing a way easily distribute
face-off plays without wasting paper and ink to do so. Coaches will also be able to save face-off plays,
allowing them to keep track of what plays they have used against who in the past. This ability to save face-off
plays will also allow coaches to save scouting reports. Every week, coaches watch video on the opposing team to 
get an idea of what they are in for come game time. While watching video, coaches can create and save the opposing 
team's faceoff plays to determine the best counter plays.  "Dot" will save coaches time and allow players to 
study their routes in an effiecent manner.}

\subsection*{Design Principles}
{Several design principles guide Dot's development to address the problem of creating and
distributing faceoff plays. With accessibiliy and usability at the forefront of priorities, Dot produces an 
SVG allowing coaches to easily distribute visual represenations of plays. Efficiency and scalability are two factors
that Dot seeks to achieve.
The ability to save and organize face-off plays supports scalability by allowing coaches to implement libraries
of stratgies/pre-scouts over time. Adaptability and extensabiliy are also fundamental design principles in the
development of the langauge. Dot allows for the creation of custom plays designed for specific opponents, providing
an advantage in the hockey world.  Furthermore, Dot's support for saving and analzying opponent's face-off plays
contributes to its stratagic use as a tool for planning and adapting. Moreover, Dot is built with accessibiliy,
efficiency, and adaptability in mind.}

\pagebreak
\subsection*{Examples}

    lefthash right offense corner\\
     righthash right offense backdoor\\
     dot right offense slot\\
     stackinside right offense walkline\\
     stackoutside right offense upwall


\vspace{0.5cm} 

     lefthash left offense backdoor\\
     righthash left offense slot\\
     dot left offense upwall\\
     stackinside left offense net\\
     stackoutside left offense walkline

\vspace{0.5cm} 

     lefthash left defense net\\
     righthash left defense walkline\\
     dot left defense net\\
     stackinside left defense slot\\
     stackoutside left defense backdoor


\subsection*{Language Concepts}

To write programs in this language, users need to have a grasp on the concepts of primitives and combining forms as 
well as knowledge of the sport of hockey. The primitives help define areas on the hockey rink through the use of side
 and zone indicators to specify the context of the location.  End routes represent endpoints 
 that players will move to from their original position once the puck drops. Combining forms involve creating these routes 
 by pairing start and end points. These routes can be organized into a board, which is a list of routes. A sequence of 
 strategic moves can then be represented off a faceoff. By understanding these core concepts, users can effectively design 
 player movements that will produce an advantage on game day.

\subsection*{Formal Syntax}
\begin{verbatim}
    <expr> ::= <route>+
    <route>::= <start><endroute>
   <start> ::= <location><dotplace>
<dotplace> ::= <side><zone>
<endroute> ::= net
            | walkline
            | downwall
            | upwall
            | corner
            | hold
            | slot
            | backdoor   
    <zone> ::= offense
            |  defense
    <side> ::= right
            |  left
  <location> ::= leftwing
            |  rightwing
            |  center
            |  rightdefense
            |  leftdefense
<location> ::= lefthash
            |  righthash
            |  dot
            |  rightpoint
            |  leftpoint
            |  stackinside
            |  stackoutside   
   \end{verbatim}

\subsection*{Semantics}

The basic building blocks of Dot are the players, locations, and end routes. Players can be left 
wing, right wing, center, etc. Possible locations include the left hash, right hash, dot, etc. End 
routes consist of options like net, walkline, corner, etc. The combined forms are made up of these basic values using different constructs such as tuples and 
variant types. For instance, I combine a player and a location to define a “position”. A “start” is 
created by combining a position with a dot place. Routes are formed by combining a start with an end 
route. Programs are run by inputting a txt. This can be done on the command line. When evaluating a program, 
an svg. file is produced with all the routes specified through the txt. file.

% DO NOT DELETE ANYTHING BELOW THIS LINE
\end{document}